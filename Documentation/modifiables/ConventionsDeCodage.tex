\documentclass{report}
\usepackage[utf8]{inputenc}     % to use a latin keyboard
\usepackage[T1]{fontenc}        % idem
\usepackage[francais]{babel}    % to say we use french language
\usepackage[top=2cm, bottom=2cm, left=2cm, right=2cm]{geometry}
\usepackage{wrapfig}
\usepackage{graphicx}
\usepackage{listings}
\usepackage{xcolor}
\usepackage{textcomp}
\usepackage{hyperref}
\lstset{frame=tblr,rulesep=1mm,framesep=2mm,framerule=1pt,xrightmargin=10mm,
        xleftmargin=20mm,
        rulecolor={\color[gray]{0.6}},
        rulesepcolor={\color[gray]{0.9}},
        texcl=true,
        literate={é}{{\'e}}1
                 {É}{{\'E}}1
				 {è}{{\`e}}1
				 {à}{{\`a}}1
				 {À}{{\`A}}1
        }
\title{Architecture logicielle\\
		\\
		Projet SOSIE2\\
		\\
		Conventions de codage}
\author{Enseignant : Marc PETIT\\
		Promotion IATIC5 (2017-2018)\\
		\\
        Institut des Sciences et Techniques des Yvelines\\
        Université de Versailles Saint-Quentin-en-Yvelines\\
		\\
        }
\begin{document}
\maketitle
\chapter*{Présentation}
Ce document a pour visée de présenter, fixer et normaliser les conventions de 
codage pour le projet SOSIE2.\\
Ce document est réalisé en \LaTeX{} et est modifiable à partir du fichier 
\begin{lstlisting}[language=c]
	$ /Documentation/modifiables/ConventionsDeCodage.tex
\end{lstlisting}
\\
Les conventions seront fixées au fur et à mesure par les équipes de développement 
des sprints. Les points possibles sont :
\begin{description}
	\item nommage des variables
	\item nommage des fonctions
	\item indentation
	\item ...
\end{description}
Une fois le document \itshape{.tex} modifié, utilisez la commande :
\begin{lstlisting}[language=c]
	$ pdflatex *.tex
\end{lstlisting}
afin de générer le document pdf à placer dans le répertoire \itshape{/Documentation}.
\end{document}                 % The input file ends with this command.
